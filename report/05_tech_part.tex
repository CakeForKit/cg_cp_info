\chapter{Технологическая часть}

В данном разделе представлены средства реализации, листинги основных алгоритмов, описаны процесс сборки, интерфейс приложения и методы тестирования.

\section{Средства реализации}
В качестве языка программирования был выбран \texttt{C++}~\cite{cpp} в силу следующих причин:
\begin{itemize}
	\item В стандартной библиотеке языка присутствует поддержка всех структур данных, выбранных по результатам проектирования;
	\item Средствами языка можно реализовать все алгоритмы, выбранные в результате проектирования;
\end{itemize}

Для создания пользовательского интерфейса был использован фреймворк~\texttt{QT}~\cite{qt}, так как данный фреймворк предоставляет инструменты для создания пользовательских интерфейсов и поддерживается язык программирования~\texttt{C++}.

Для сборки программного обеспечения использовалась утилита~\texttt{CMake}~\cite{cmake}, так как с ee помощью возможно управлять процессом компиляции и сборки проекта, написанного на \texttt{C++}.

Для модульного тестирования компонент программного обеспечения был выбран фреймворк~\texttt{GoogleTest}~\cite{gtest}, так как данный фреймворк предоставляет инструменты для написания модульных тестов на языке~\texttt{C++}.

Для определения покрытия кода использовалась утилита~\texttt{gcov}~\cite{gcov}.

\clearpage
\section{Реализации алгоритма обратной трассировки лучей}

Для генерации кадра, состоящего из шахматных фигур, заданных полигонами, и шахматной доски с глянцевой или матовой поверхностью был использован алгоритм обратной трассировки лучей с глобальной моделью освещения. В этом методе для каждого пикселя генерируемого изображения создается луч, исходящий от наблюдателя, и для данного луча запускается алгоритм трассировки луча, представленный в приложении А. 

Для поданного на вход луча определяется его пересечение с ближайшим видимым объектом сцены, в случае отсутствия пересечения, пиксель, определяемый лучом, закрашивается цветом фона. Если пересечение найдено, запускается луч, направленный на источник света для определения освещенности точки в пространстве, и запускается луч отражения если поверхность пересечения является глянцевой.

%В листинге~\ref{lst:castRay} представлена реализация алгоритма трассировки луча.
\clearpage
\section{Описание процесса сборки приложения}
Для сборки программного обеспечения использовалась утилита~\texttt{CMake}~\cite{cmake}. Для сборки приложения необходимо в командной строке, находясь в директории проекта, выполнить следующие команды~\ref{lst:build}.

\begin{center}
	\captionsetup{justification=raggedright,singlelinecheck=off}
	\renewcommand{\lstlistingname}{Листинг}
	\begin{lstlisting}[label=lst:build, caption=Сборка программного обеспечения]
		$ cd build
		$ cmake -S ..
		$ cmake -build .
	\end{lstlisting}
\end{center}
\clearpage

\section{Описание интерфейса приложения}
На рисунках~\ref{img:interface_create}-\ref{img:interface_tab} представлен интерфейс программы. 

На рисунке~\ref{img:interface_create} представлена вкладка добавления фигуры определенного игрока из выпадающего списка на позицию шахматной доски, изменение цветового набора и изменения материала шахматной доски. На рисунке~\ref{img:interface_move} представлена вкладка изменения позиции и вращения модели с индексом, определенным на вкладке~\ref{img:interface_tab}. На рисунке~\ref{img:interface_camera} представлена вкладка изменения позиции и вращения камеры. На рисунке~\ref{img:interface_scene} представлена вкладка изменения содержимого сцены. На рисунке~\ref{img:interface_scene} представлена вывода информации о сцене.

\FloatBarrier
\imgw{0.5\textwidth}{interface_create}{Интерфейс программы. Вкладка добавления моделей и настроек материалов}
\FloatBarrier
\imgw{0.5\textwidth}{interface_move}{Интерфейс программы. Вкладка изменения позиции модели}
\FloatBarrier
\imgw{0.5\textwidth}{interface_camera}{Интерфейс программы. Вкладка изменения позиции камеры}
\FloatBarrier
\imgw{0.5\textwidth}{interface_scene}{Интерфейс программы. Вкладка настроек сцены}
\FloatBarrier
\imgw{0.5\textwidth}{interface_tab}{Интерфейс программы. Вкладка вывода информации о сцене}
\FloatBarrier

\clearpage
\section{Тестирование}

\subsection{Функциональное тестирование}
Для функционального тестирования была рассмотрена сцена, на которой изображены все виды фигур, и выделены следующие классы эквивалентности: камера расположена под углом $30^\circ$~(\ref{img:ft_3}); вид сбоку~(\ref{img:ft_2}); вид сверху~(\ref{img:ft_1}).

\FloatBarrier
\imgw{0.45\textwidth}{ft_3}{Камера расположена под углом $30^\circ$}
\FloatBarrier
\imgw{0.45\textwidth}{ft_2}{Вид сбоку}
\FloatBarrier
\imgw{0.45\textwidth}{ft_1}{Вид сверху}
\FloatBarrier

\subsection{Модульное тестирование}
Для модульного тестирования компонент программного обеспечения использовался фреймворк~\texttt{GoogleTest}~\cite{gtest}. 

Были созданы параметризованные тесты для функций классов: $Camera$, $RayTracing$, $Ray$ и $Triangle$. Для тестов каждого рассматриваемого объекта использовался класс фиксации, в котором производились настройки окружения, например создание классов $SceneManager$ и $MaterialManager$, в функции SetUp, а также освобождение памяти, при необходимости, в функции TearDown.


\subsection{Результаты тестирования}

В качестве меры полноты тестирования был выбран процент покрытия строк кода. Результаты тестирования, были обработаны утилитой~\texttt{lcov}~\cite{lcov} и представленный в таблице~\ref{tbl:resTest}:

\begin{longtable}{|
		>{\centering\arraybackslash}m{.3\textwidth - 2\tabcolsep}|
		>{\centering\arraybackslash}m{.3\textwidth - 2\tabcolsep}|
		>{\centering\arraybackslash}m{.4\textwidth - 2\tabcolsep}|
	}
	\caption{Результаты тестирования}\label{tbl:resTest} \\\hline
	Количество протестированных строк кода & Количество строк кода в проекте & Процент покрытия для созданного набора тестов \\\hline
	\endfirsthead
	\caption*{Продолжение таблицы~\ref{tbl:log} } \\\hline
	Количество протестированных строк кода & Количество строк кода в проекте & Процент покрытия для созданного набора тестов \\\hline \\\hline                    
	\endhead
	\endfoot
	2376 & 3096 & 77 \% \\\hline
%	Functions & 220 & 463 & 47.5 \% \\\hline
	
\end{longtable}

%\clearpage
\section*{Вывод из технологической части}
Было реализовано программное обеспечение и представлены средства реализации, описание алгоритма обратной трассировки лучей, описаны интерфейс приложения и методы тестирования.
\clearpage
