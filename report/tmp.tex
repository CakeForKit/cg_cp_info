\chapter{Аналитическая часть}

В данном разделе проводится глубокий анализ задачи построения трёхмерного изображения сцены, рассматриваются различные методы и алгоритмы, решающие данную задачу. Обосновывается выбор реализуемого алгоритма и указываются ограничения, в рамках которых будет работать разрабатываемое программное обеспечение.

%\section{Описание модели трёхмерного объекта на сцене}
%
%Трёхмерные модели могут задаваться различными способами \cite{rodgers, rogers_book}:
%
%\begin{enumerate}[label=\arabic*)]
%	\item \textbf{Каркасная (проволочная) модель.} В этой модели задается информация о вершинах и рёбрах объектов. Это простейший вид моделей. Недостатком является невозможность отличить видимые грани от невидимых.
%	\item \textbf{Поверхностная модель.} Поверхность может описываться аналитически или другим способом. Недостаток заключается в отсутствии информации о том, с какой стороны поверхности находится материал.
%	\item \textbf{Объёмная модель.} Отличается от поверхностной тем, что включает информацию о том, с какой стороны поверхности расположен материал. Это позволяет более точно моделировать взаимодействие света с объектом.
%\end{enumerate}
%
%Для решения поставленной задачи не подходит каркасная модель из-за невозможности различить видимые и невидимые грани, что критично для построения реалистичного изображения. Поверхностная модель также не подходит, так как необходимо учитывать свойства материала с обеих сторон поверхности. Таким образом, выбор пал на объёмную модель.
%
%\section{Описание способа задания трёхмерного объекта на сцене}
%
%Существует несколько способов задания объёмной модели \cite{rodgers, rogers_book}:
%
%\begin{enumerate}[label=\arabic*)]
%	\item \textbf{Аналитический способ.} Характеризуется описанием объекта в неявной форме, требуя вычисления значений функции в различных точках пространства для получения визуального представления.
%	\item \textbf{Полигональная сетка.} Совокупность вершин, рёбер и граней, обычно треугольников, определяющих форму многогранного объекта. Существуют различные способы хранения информации о полигональной сетке:
%	\begin{enumerate}[label=\arabic*)]
%		\item \textbf{Список граней.} Хранит множество граней и вершин, каждая грань состоит минимум из трёх вершин.
%		\item \textbf{<<Крылатое>> представление.} Включает вершины, грани и рёбра сетки, с дополнительными индексами, что увеличивает объем памяти и усложняет генерацию списка граней.
%		\item \textbf{Вершинное представление.} Описывает объект как множество соединённых вершин без явного указания граней и рёбер, что усложняет генерацию списка граней при рендеринге.
%	\end{enumerate}
%\end{enumerate}
%
%При выборе способа задания объекта ключевым фактором стала скорость выполнения геометрических преобразований. Оптимальным представлением является полигональная сетка с хранением в виде списка граней, что обеспечивает явное описание граней и упрощает выполнение геометрических преобразований без необходимости генерации дополнительных списков.
%
%\section{Формализация объектов сцены}

Сцена состоит из следующих объектов:

\begin{enumerate}[label=\arabic*)]
	\item \textbf{Геометрический объект.} Представляется в виде полигональной сетки, требующей указания координат вершин, связей между вершинами (рёбер), фонового, диффузного и зеркального освещения (цветов), коэффициентов освещения, степени зеркального отражения, коэффициентов отражения и преломления, а также показателя преломления.
	\item \textbf{Источник света.} Представляется как точечный объект с характеристиками расположения, цвета и интенсивности излучения.
	\item \textbf{Камера.} Характеризуется пространственным положением и направлением взгляда.
\end{enumerate}

\section{Анализ алгоритмов удаления невидимых линий и поверхностей}

Построение трёхмерного изображения сцены заключается в преобразовании объектов сцены в изображение на растровом дисплее. Для создания реалистичного изображения необходимо учитывать несколько факторов:

\begin{enumerate}[label=\arabic*)]
	\item Невидимые линии и поверхности.
	\item Тени.
	\item Освещение.
	\item Свойства материалов объекта: способность отражать и преломлять свет.
\end{enumerate}

Рассмотрим основные алгоритмы удаления невидимых линий и поверхностей:

\subsection{Алгоритм Робертса}

Алгоритм Робертса работает в объектном пространстве и предназначен для решения задачи удаления невидимых линий для исключительно выпуклых тел \cite{rogers, rogers_book}.
\textbf{Этапы алгоритма:}


\begin{enumerate}[label=\arabic*)]
	\item \textbf{Подготовка исходных данных.} Формируется матрица тела $V$, где каждый столбец содержит коэффициенты уравнения плоскости грани.
	\item \textbf{Удаление ребер, экранируемых самим телом.} Используется вектор взгляда для определения невидимых граней.
	\item \textbf{Удаление невидимых ребер, экранируемых другими телами.} Построение лучей от наблюдателя к ребру и проверка пересечения с другими телами.
\end{enumerate}

\textbf{Свойства алгоритма Робертса:}
\begin{enumerate}[label=\arabic*)]
	\item Высокая точность благодаря работе в объектном пространстве.
	\item Квадратичная сложность по числу объектов.
	\item Поддержка только выпуклых тел.
\end{enumerate}

\textbf{Недостатки:}
\begin{enumerate}[label=\arabic*)]
	\item Проблемы при наличии невыпуклых тел.
	\item Невозможность визуализации зеркальных поверхностей.
\end{enumerate}

Таким образом, алгоритм Робертса не подходит для поставленной задачи из-за своих ограничений.

\subsection{Алгоритм Z--буфера}

Алгоритм Z--буфера работает в пространстве изображений и использует два буфера: буфер кадра и Z--буфер для хранения глубины каждого пикселя \cite{rogers, rogers_book}.

\textbf{Этапы алгоритма:}
\begin{enumerate}[label=\arabic*)]
	\item Инициализация Z--буфера минимально возможными значениями $z$.
	\item Преобразование каждого многоугольника в растровую форму и запись в буфер кадра.
	\item Сравнение глубины новых пикселей с текущими в Z--буфере и обновление при необходимости.
	\item Вычисление глубины $z$ для каждого пикселя на основе уравнения плоскости.
\end{enumerate}

\textbf{Свойства алгоритма Z--буфера:}
\begin{enumerate}[label=\arabic*)]
	\item Линейная сложность.
	\item Большой объём требуемой памяти.
	\item Сложность реализации эффектов прозрачности и зеркальности.
	\item Дополнительные вычислительные операции для невыпуклых тел.
\end{enumerate}

\textbf{Недостатки:}
\begin{enumerate}[label=\arabic*)]
	\item Трудности с визуализацией прозрачных и зеркальных поверхностей.
\end{enumerate}

Таким образом, алгоритм Z--буфера также не подходит для решения поставленной задачи из-за своих ограничений.

\subsection{Алгоритм обратной трассировки лучей}

Алгоритм обратной трассировки лучей отслеживает лучи от наблюдателя к объектам сцены, повышая эффективность по сравнению с прямой трассировкой лучей \cite{shykin, bayackovskiy}.

\textbf{Этапы алгоритма:}
\begin{enumerate}[label=\arabic*)]
	\item Преобразование сцены в пространство изображения.
	\item Испускание лучей от наблюдателя через пиксели растра к сцене.
	\item Проверка пересечения лучей с объектами сцены.
	\item Определение видимых поверхностей по глубине пересечений.
	\item Рекурсивное отражение и/или преломление лучей при наличии отражающих или прозрачных материалов.
	\item Учёт теней путём проверки видимости световых источников из точки пересечения.
\end{enumerate}

\textbf{Свойства алгоритма обратной трассировки лучей:}
\begin{enumerate}[label=\arabic*)]
	\item Высокая реалистичность синтезируемого изображения.
	\item Учет теней и эффектов отражения и преломления.
	\item Возможность визуализации зеркальных и прозрачных поверхностей.
	\item Увеличенное время выполнения из-за рекурсивных вычислений.
\end{enumerate}

\textbf{Недостатки:}
\begin{enumerate}[label=\arabic*)]
	\item Высокие вычислительные затраты, особенно при сложных сценах.
\end{enumerate}

\subsection{Другие алгоритмы растеризации}

Стоит отметить, что другие алгоритмы растеризации, включая алгоритмы Робертса и Z--буфера, не способны эффективно визуализировать отражающие поверхности \cite{rodgers}. Поэтому дальнейшее рассмотрение подобных алгоритмов нецелесообразно для поставленной задачи.

\subsection{Вывод по алгоритмам удаления невидимых линий и поверхностей}

На основе проведенного анализа наиболее подходящим для решения поставленной задачи является алгоритм обратной трассировки лучей благодаря высокой реалистичности изображения и возможности визуализации отражающих и прозрачных поверхностей, несмотря на его высокую вычислительную сложность.

\section{Учёт теней}

При использовании алгоритма обратной трассировки лучей построение теней осуществляется автоматически за счёт проверки видимости источников света из точек пересечения лучей с объектами. Если луч от точки пересечения к источнику света пересекает другой объект, точка считается находящейся в тени.

\section{Учёт освещения}

Модель освещения играет ключевую роль в создании реалистичного изображения \cite{rodgers, rogers_book}. Выбран глобальный подход, который учитывает не только прямое освещение от источников, но и освещение, отражённое от других объектов сцены.

\textbf{Формула интенсивности освещения:}
\begin{equation}
	I = k_0I_0 + k_d\sum\limits_{j}^{}I_j(n \cdot l_j) + k_r\sum\limits_{j}^{}I_j(s \cdot r_j)^{\beta} + k_rI_r + k_tI_t,
\end{equation}
где:
\begin{itemize}
	\item $k_0$ — коэффициент фонового освещения,
	\item $k_d$ — коэффициент диффузного отражения,
	\item $k_r$ — коэффициент зеркального отражения,
	\item $k_t$ — коэффициент пропускания,
	\item $n$ — нормаль к поверхности в точке,
	\item $l_j$ — вектор к $j$-му источнику света,
	\item $s$ — вектор к наблюдателю,
	\item $r_j$ — отражённый вектор $l_j$,
	\item $I_0$ — интенсивность фонового освещения,
	\item $I_j$ — интенсивность $j$-го источника света,
	\item $I_r$ — интенсивность отражённого света,
	\item $I_t$ — интенсивность преломлённого света.
\end{itemize}

Глобальная модель освещения позволяет учитывать сложные взаимодействия света с объектами сцены, такие как зеркальное отражение и преломление, что необходимо для достижения высокого уровня реалистичности изображения.

\section{Выводы из аналитической части}

Были рассмотрены различные способы задания трёхмерных моделей и выбрана объёмная модель с использованием полигональной сетки и списка граней для эффективного выполнения геометрических преобразований.

Проанализированы основные алгоритмы удаления невидимых линий и поверхностей:
\begin{enumerate}[label=\arabic*)]
	\item Алгоритм Робертса,
	\item Алгоритм Z--буфера,
	\item Алгоритм обратной трассировки лучей.
\end{enumerate}

В качестве оптимального решения для данной задачи выбран алгоритм обратной трассировки лучей с глобальной моделью освещения, обеспечивающий высокую реалистичность синтезируемого изображения и поддерживающий визуализацию отражающих и прозрачных поверхностей.

\newpage
\clearpage