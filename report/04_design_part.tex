\chapter{Конструкторская часть}

В данном разделе представлены математические основы и схема алгоритма обратной трассировки лучей, функциональная модель и структура программного обеспечения. 
%Так же будут описаны используемые структуры данных и структура реализуемого программного обеспечения.


\section{Математические основы алгоритма обратной трассировки лучей}
\subsection{Определение пересечения луча с полигоном}
Так как любую плоскость можно однозначно задать тремя точками, для поиска пересечения луча с полигонами используется алгоритм Моллера~--~Трумбора~\cite{trIntersect}, с помощью которого можно вычислить пересечение луча с треугольным полигоном. Рассмотрим данн

Пусть треугольный полигон определен вершинами $V_0, V_1, V_2$, а луч $R(t)$ с началом в точке $O$ и единичным  вектором направления $D$ определен формулой
\begin{equation}
	R(t) = O + tD
\end{equation}

И если точку $T(u,v)$ на треугольнике $V_0V_1V_2$ выразить через ее барицентрические координаты $(u,v)$, так что ($u \geq 0, v \geq 0, u + v \leq 1$):
\begin{equation}
	T(u,v) = (1 - u - v)V_0 + uV_1 + vV_2,
\end{equation}
тогда пересечение луча $R(t)$ и треугольника $V_0V_1V_2$ эквивалентно решению уравнения
$R(t) = T(u, v)$ и однозначно определяются параметрами расстояния $t$ от начала луча до точки пересечения и барицентрическими координатами $(u,v)$. В таком случае получим:
\begin{equation}\label{eq:eq1}
	O + tD = (1- u - v)V_0 + uV_1 + vV_2
\end{equation}

Уравнение~\ref{eq:eq1} может быть представнено в матричном виде:
\begin{equation}
	\label{slau}
	\begin{bmatrix}
		-D & V_1 - V_0, V_2 - V_0
	\end{bmatrix}
	\begin{bmatrix}
		t\\
		u\\
		v
	\end{bmatrix} = O - V_0 
\end{equation}

Пусть $E_1 = V_1 - V_0, E_2 = V_2 - V_0$, $T=O - V_0$. Решение уравнения~\ref{eq:eq1} можно получить методом Крамера:
\begin{equation}\label{eq:eq2}
	\begin{bmatrix}
		t\\
		u\\
		v
	\end{bmatrix} = \frac{1}{(D\times E_2) \cdot E_1}
	\begin{bmatrix}
		(T\times E_1) \cdot E_2\\
		(D\times E_2) \cdot T\\
		(T\times E_1) \cdot D
	\end{bmatrix}
\end{equation}
 
Если баричентически координаты точки пересечения полученной из формулы~\ref{eq:eq2} удовлетворяют условию ($u \geq 0, v \geq 0, u + v \leq 1$), то луч $R(t)$ пересекает треугольный полигон заданный вершинами $V_0, V_1, V_2$. Если определитель $(D\times E_2) \cdot E_1$ равен нулю то луч лежит в плоскости треугольника $V_0V_1V_2$.


%\subsection{Определение нормали к плоскости}
%Нормаль к плоскости можно найти как векторное произведение двух векторов, принадлежащих данной плоскости~\cite{rodgersCG}:
%
%Для поиска нормали к полигонам необходимо найти векторное произведение двух векторов, которые лежат на полигоне \cite{rodgers}.
%\begin{equation}
%	N = (V_2 - V_0) \times (V_1 - V_0),
%\end{equation}
%где $V_0, V_1, V_2$ -- вершины полигона.

\subsection{Определение вектора отраженния}

Для визуализации отражающих поверхностей в алгоритме обратной трассировки лучей необходим способ определения направления вектора отражения зная луч падения~$l$ и нормаль к поверхности~$n$~\cite{rtOneWeekend}.

\imgw{0.4\textwidth}{reflect}{Расчет направления вектора отражения}

Вектор отражения представляется через разность вектора падения $l$ и вектора нормали $n$, длина которого равняется длине двух проекций вектора $l$ на $n$:

\begin{equation}
	\label{reflect_ray}
	R = l - 2 n \cdot \frac{(l, n)}{(n, n)}
\end{equation}

\clearpage
\section{Функциональная модель программного обеспечения}
 idef0

\clearpage
\section{Описание алгоритма обратной трассировки лучей}
На рисунках~\ref{img:ray_tracing}-~\ref{img:castRay} представлены схемы алгоритма испускания луча и алгоритма обратной трассировки лучей.
\FloatBarrier
\imgh{0.6\textheight}{ray_tracing}{Схема алгоритма обратной трассировки лучей}
\FloatBarrier
\imgh{0.9\textheight}{castRay}{Схема алгоритма испускания луча}
\FloatBarrier


\clearpage
\section{Структура разрабатываемого программного обеспечения}
На рисунках~\ref{img:facade}-~\ref{img:loadManager} представлена диаграмма классов разрабатываемого программного обеспечения.
\FloatBarrier
\imgw{0.9\textwidth}{facade}{Диаграмма классов}
\FloatBarrier
\imgw{1\textwidth}{sceneManager}{Диаграмма классов}
\FloatBarrier
\imgw{1\textwidth}{loadManager}{Диаграмма классов}
\FloatBarrier

%\section{Описание используемых типов и структур данных}
%\section{Требования к программному обеспечению}

\section*{Вывод}

В данном разделе были представлены математические основы алгоритма обратной трассировки лучей и спроектировано програмное обеспечение, которое было описано функциональной моделю, схемой алгоритма обратной трассировки лучей и структурой, представленной в виде диаграммы классов.

\clearpage
