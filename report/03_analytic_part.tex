\chapter{Аналитическая часть}

%В данном разделе рассматриваются алгоритмы построения реалистичных изображений и способы представления трехмерных моделей в программе. Также обосновывается сделанный выбор.

В данном разделе проводится анализ задачи построения трёхмерного реалистичного изображения, рассматриваются методы, решающие данную задачу, обосновывается выбор реализуемого алгоритма и указываются ограничения, в рамках которых будет работать разрабатываемое программное обеспечение.


\section{Выбор алгоритма построения реалистичных изображений}

Для создания реалистичного изображения необходимо учитывать такие факторы как невидимые линии и поверхности, тени, освещение, свойства материалов объекта (способность отражать и преломлять свет).


\subsection{Анализ алгоритмов удаления невидимых ребер и поверхностей}
\subsection*{Алгоритм Робертса}

Алгоритм Робертса решает задачу удаления невидимых линий и работает в объектном пространстве исключительно с выпуклыми телами, если тело является не выпуклым, то его нужно разбить на выпуклые составляющие~\cite{rodgersCG}.

Этапы алгоритма:
\begin{enumerate}[label=\arabic*)]
	\item \textbf{Подготовка исходных данных.} В данном алгоритме выпуклое многогранное тело представляется набором пересекающихся плоскостей. Формируется матрица тела~$V$, где каждый столбец содержит коэффициенты уравнения плоскости грани~$ax + by + cz + d = 0$:
	\begin{equation}
		V = \begin{pmatrix}
			a_{1} & a_{2} & \ldots & a_{n}\\
			b_{1} & b_{2} & \ldots & b_{n}\\
			c_{1} & c_{2} & \ldots & c_{n}\\
			d_{1} & d_{2} & \ldots & d_{n}
		\end{pmatrix}.
	\end{equation}
	\item \textbf{Удаление ребер, экранируемых самим телом.} Используется вектор взгляда~$E = (0, 0, -1, 0)$ для определения невидимых граней. При умножении вектора $E$ на матрицу тела $V$ отрицательные компоненты полученного вектора будут соответствовать невидимым граням;
	
	\item \textbf{Удаление невидимых ребер, экранируемых другими телами.} Для определения невидимых точек ребра строится луч, соединяющий точку наблюдения с точкой на ребре. Точка невидима, если луч на своём пути встречает в качестве преграды тело, т.е. проходит через него;
\end{enumerate}

\textbf{Преимущества алгоритма Робертса:}
\begin{itemize}
	\item Высокая точность благодаря работе в объектном пространстве;
\end{itemize}

\textbf{Недостатки алгоритма Робертса:}
\begin{itemize}
	\item Не работает с невыпуклыми телами;
	\item Невозможность визуализации отражающих поверхностей;
%	\item Квадратичная сложность по числу объектов;
\end{itemize}


\subsection*{Алгоритм, использующий z-буфер}
Алгоритм Z--буфера работает в пространстве изображений и использует два буфера: буфер кадра для хранения цвета каждого
пикселя и Z--буфер для хранения глубины каждого пикселя \cite{shikinStCG}.

\textbf{Этапы алгоритма:}
\begin{enumerate}[label=\arabic*)]
	\item Инициализация Z--буфера минимально возможными значениями и буфера кадра значениями пикселя, описывающими фон;
	\item Преобразование каждой проекции грани многоугольников в растровую форму;
	\item Вычисление для каждого пикселя с координатами $(x, y)$, его глубины $Z(x, y)$.
	\item Сравнение глубины $Z(x, y)$ новых пикселей с текущими в Z--буфере и обновление буфера кадра при необходимости;
\end{enumerate}

\textbf{Преимущества алгоритма Z--буфера:}
\begin{itemize}
%	\item Время работы алгоритма не зависит от разрешения обьекта сцены (числа граней поверхности многогранника)
%	\item Линейная сложность по числу точек растра и усредненного числа граней поверхности многогранника;
	\item Простота алгоритма и используемого в нем набора операций;
\end{itemize}

\textbf{Недостатки алгоритма Z--буфера:}
\begin{itemize}
	\item Большой объём требуемой памяти;
	\item Невозможность визуализации отражающих поверхностей;
\end{itemize}

Таким образом, алгоритм Z--буфера также не подходит для решения поставленной задачи из-за своих ограничений.

\subsection*{Алгоритм обратной трассировки лучей}
Алгоритм обратной трассировки лучей заключается в отслеживии траектории лучей, которые исходят из точки наблюдателя и проходят через центр пикселя растра в направлении сцены. \cite{shikinDinamica}.

\textbf{Этапы алгоритма:}
\begin{enumerate}[label=\arabic*)]
	\item Преобразование сцены в пространство изображения, т. е. область видимости наблюдателя разбивается на пиксели.
	\item Испускание лучей от наблюдателя через пиксели растра к сцене.
	\item Определение ближайшего пересечения лучей с объектами сцены.
	\item Определение находится ли точка пересечения в тени путем испускания луча из этой точки в направлении источника света. И если луч пересекает какие-либо объекты сцены, то точка находится в тени.
	\item Рекурсивное отражение и/или преломление лучей при наличии отражающих или прозрачных материалов.
	\item Учёт теней путём проверки видимости световых источников из точки пересечения.
\end{enumerate}

\textbf{Преимущества алгоритма обратной трассировки лучей:}
\begin{enumerate}[label=\arabic*)]
	\item Высокая реалистичность синтезируемого изображения.
	\item Учет теней и эффектов отражения и преломления.
\end{enumerate}

\textbf{Недостатки алгоритма обратной трассировки лучей:}
\begin{enumerate}[label=\arabic*)]
	\item Увеличенное время выполнения из-за рекурсивных вычислений.
\end{enumerate}

\subsection*{Вывод}
В качестве алгоритма удаления невидимых ребер и поверхностей был выбран алгоритм обратной трассировки лучей, так как с его помощью возможно реализовать эффект отражения, необходимый для решения поставленной задачи. А так же данный алгоритм не требует дополнительной реализации алгоритмов закраски и построения теней.

\subsection{Анализ модели освещения}

Модель освещения определяет цвет поверхности обьекта отображаемого на экране. Модель освещения бывает двух видов: локальная и глобальнаяю. В локальной модели освещения учитывается только свет падающий от источников и ориентация поверхности в пространстве. В глобальной модели освещения дополнительно еще учитывается свет, отражённый от других поверхностей и/или пропущенный через них.

Так как для решения поставленной задачи необходимо реализовать отражающую поверхность шазматной доски, в работе использовалась глобальная модель освещения, которая являлась состояной частью алгоритма обратной трассировки лучей.




%\section{Описание представления трехмерных моделей на сцене}
%
%\section{Формализация объектов сцены}


%\section*{Вывод}
%В данном разделе были теоретически разобраны два алгоритмы умножения матриц: стандартного и Винограда.


\clearpage
