\chapter{Исследовательская часть}

В данном разделе описано исследование зависимости времени генерации кадра от числа полигонов на сцене для варьируемого числа рабочих потоков.

\section{Технические характеристики}
\begin{itemize}
	\item {Гостевая операционная система --- Ubuntu 22.04.4 LTS}
	\item {Установленная оперативная память (RAM) -- 8,0 ГБ;}
	\item {Процессор -- AMD Ryzen 7 5800H with Radeon Graphics, 3201 МГц, ядер: 8, логических процессоров: 16;}
\end{itemize}

При проведении замеров времени ноутбук был включен в сеть электропитания, а не работал от аккамулятора, и были запущены только встроенное приложение окружения и система замеров времени.

\section{Цель исследования}
Цель исследования -- сравнительный анализ зависимости времени генерации кадра от числа полигонов на сцене для варьируемого числа рабочих потоков.

\section{Описание исследования}

Был реализован алгоритм параллельной генерации кадра, для этого использовалась функция создания потоков~\texttt{pthread\_create}~\cite{thread-city}. Каждый поток в данной реализации отвечал за определение цвета группы пикселей методом обратной трассировки лучей. 

В ходе исследования были проведены замеры реального времени генерации кадра, для этого использовалась функция ~\texttt{gettimeofday}~\cite{time-city}. Было рассмотрено время отрисовки сцены для дополнительных рабочих потоков в размере от 0 (вычисление в основном потоке) до 64, по степеням числа 2, то есть было рассмотрено 0, 1, 2, 4, 8, 16, 32, 64 дополнительных рабочих потока.

Замеры времени проводились по принципу: для одних входных данных проводилось 5 замеров и если относительная стандартная ошибка среднего (rse) была >= 5\%, то для этих данных замеры продолжались, в таблицу заносилось среднее значение.

\section{Результат исследования}
Результаты замеров времени с секундах приведены в таблице~\ref{tbl:timeData}

\begin{longtable}{|
		>{\centering\arraybackslash}m{.2\textwidth - 2\tabcolsep}|
		>{\centering\arraybackslash}m{.1\textwidth - 2\tabcolsep}|
		>{\centering\arraybackslash}m{.1\textwidth - 2\tabcolsep}|
		>{\centering\arraybackslash}m{.1\textwidth - 2\tabcolsep}|
		>{\centering\arraybackslash}m{.1\textwidth - 2\tabcolsep}|
		>{\centering\arraybackslash}m{.1\textwidth - 2\tabcolsep}|
		>{\centering\arraybackslash}m{.1\textwidth - 2\tabcolsep}|
		>{\centering\arraybackslash}m{.1\textwidth - 2\tabcolsep}|
		>{\centering\arraybackslash}m{.1\textwidth - 2\tabcolsep}|
	}
	\caption{Замеры времени отрисовки сцены в секундах от числа полигонов на сцене для варьируемого числа рабочих потоков}\label{tbl:timeData} \\\hline
	Количество полигонов & \multicolumn{8}{c|}{\centering Количество потоков} \\ \cline{2-9}
	& 0 & 1 & 2 & 4 & 8 & 16 & 32 & 64 \\ \hline
	\endfirsthead
	\caption*{Продолжение таблицы~\ref{tbl:timeData} } \\\hline
	Количество полигонов & \multicolumn{8}{c|}{\centering Количество потоков} \\ \cline{2-9}
	& 0 & 1 & 2 & 4 & 8 & 16 & 32 & 64 \\ \hline                
	\endhead
	\endfoot
	100 & 8.34 & 12.48 & 10.53 & 11.26 & 11.50 & 10.64 & 10.18 & 8.36 \\ \hline
	200 & 17.83 & 18.26 & 15.66 & 14.06 & 12.53 & 11.71 & 11.47 & 9.25 \\ \hline
	300 & 23.28 & 21.86 & 17.76 & 16.34 & 14.69 & 13.61 & 12.86 & 10.32 \\ \hline
	400 & 38.25 & 31.23 & 24.48 & 19.60 & 17.99 & 16.69 & 15.76 & 12.25 \\ \hline
	500 & 47.95 & 37.93 & 29.48 & 22.94 & 20.95 & 19.95 & 18.92 & 15.45 \\ \hline
\end{longtable}

График зависимости времени отрисовки сцены от числа полигонов на сцене для варьируемого числа рабочих потоков представлен на рисунке~\ref{img:graph}.

\FloatBarrier
\imgw{0.9\textwidth}{graph}{Зависимость времени отрисовки сцены от числа полигонов на сцене для варьируемого числа рабочих потоков}
\FloatBarrier

Из проведённых замеров можно сделать следующие выводы:
\begin{itemize}
	\item {при увеличении количества потоков от 1 до 32 время обработки сцены уменьшается, при условии что на сцене 200 полигонов и более, так как в таком случае время необходимое для создания и запуска дополнительных потоков, равно или больше времени отрисовки всей сцены этими потоками;}
	\item {при небольшом заполнении сцены (количество полигонов = 100) алгоритм использующий последовательную обработку данных работает быстрее, так как не тратит время на создание потоков;}
	\item {при работе 64 потоков и более время генерации кадра не уменьшается, так как затраты времени на создание и запуск потоков становятся слишком большими.}
\end{itemize}


\section{Вывод из исследовательской части}
В данном разделе было проведено исследование зависимости времени генерации кадра от числа полигонов на сцене для варьируемого числа рабочих потоков и в результате сравнительного анализа были сделаны выводы о том, что при количестве полигонов на сцене меньшем 200, эффективнее по времени использовать алгоритм обратной трассировки лучей с одним потоком, а при  количестве полигонов большем 200 эффективнее использовать 32 дополнительных рабочих потока.


\clearpage
