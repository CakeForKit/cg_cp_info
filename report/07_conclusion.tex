\ssr{ЗАКЛЮЧЕНИЕ}

В ходе лабораторной работы была выполнена поставленнная цель, которая заключалась в выполнении оценки ресурсной эффективности алгоритмов умножения матриц и их реализации.

Были выполнены следующие задачи:
\begin{enumerate}
	\item Описать математическую основу стандартного алгоритма и алгоритма Винограда умножения матриц;
	\item Описать модель вычислений;
	\item Разобрать алгоритм умножения матриц -- стандартный, Винограда, оптимизированный согласно варианту алгоритм Винограда;
	\item Выполнить оценку трудоемкости разработанных алгоритмов либо их реализации;
	\item Реализовать разработанные алгоритмы в програмном обеспечении с 2 режимами работы -- одиночного расчета и массированного замера процессорного времени выполнения реализации каждого алгоритма;
	\item Выполнить замеры процессорного времени выполнения реализации разработанных алгоритмов в зависимости от варьируемого размера матриц;
	\item Выполнить сравнительный анализ рассчитанных трудоемкости и результатов замера процессорного времени выполнения реализации трех алгоритмов с учетом лучшего и худшего случаев по трудоемкости;
\end{enumerate}


Основываясь на проведенном исследовании можно сделать следующие выводы.
\begin{itemize}
	\item{Оптимизированный алгоритм Винограда демонстрирует наилучшие результаты по времени работы на всех тестовых данных, как в расчитанной ранее формула, так и на практике;}
	\item{Как и ожидалось умножение матрицы нечетного размера требует больше времени;}
	\item{На практике стандартный алгоритм умножения матриц работает медленнее, чем неоптимизированный алгоритм Винограда, это можно обьяснять оптимизацией компилятора или неточность введенной модели вычислений;}
\end{itemize}
